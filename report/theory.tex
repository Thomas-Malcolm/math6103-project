

\section{Theory}

\subsection{Assumptions}

We will somewhat unconventionally begin by stating what our assumptions are in our model, as we will use this 
to guide the theory that we develop by making specific reference to them. We will assume:
\begin{itemize}
    \item All neutrons have the same kinetic energy
    \item Our medium will be a single isotope solid that is isotropic in density. This material will 
    be either Uranium-235 or Plutonium-239
    \item Our medium's geometry will be perfectly spherical
    \item All neutrons after moving will have an event occur, which will be either an elastic collision, a capture collision, 
    or a fission reaction. Notably, we are ignoring the case of in-elastic collisions
    \item We only consider $(n,2n)$ fission reactions
    \item There is a single point neutron source at the centre of the sphere which only emits neutrons at the start of the 
    simulation
\end{itemize}
What each of these means we will touch on as we come across them, though for now take them for granted. After introducing 
the theory for them, we will discuss the implications of making these assumptions.

\subsection{Nuclear Fission}

While nuclear fission can occur without the aid of neutron bombardment (via a process known as ``spontaneous fission''), here 
we will concern ourselves exclusively with the case of a fissile material being subject to a population of energetic neutrons. 

\subsubsection{Fission Process}

When an energised neutron collides with a fissile material such as Uranium-235 or Plutonium-239, 

$$n + \prescript{235}{}U \to \prescript{236}{}U^{*} \to \prescript{134}{}Xe + \prescript{100}{}Sr + 2n$$
$$n + \prescript{239}{}Pu \to \prescript{240}{}Pu^{*} \to \prescript{134}{}Xe + \prescript{103}{}Zr + 3n$$

The neutron is initially absorbed by the Uranium / Plutonium nucleus, however, both of these elements become 
unstable with the extra mass. One of the potential outcomes of this instability is that the atom splits into two, 
and in the process of doing so, extra neutrons are emitted at high energy. Alongside this a large quantity of energy is 
released, which is what is harnessed in nuclear reactors (and provides that explosive power in a bomb). 

\noindent \textbf{Criticality} \\
Criticality with respect to nuclear fission reactions describes whether or not a system is self-sustaining in its reaction. There 
are a couple of associated terms:\\
\noindent\textit{Super-critical}: This is the state of a self-sustaining fission reaction which continues to grow in strength\\
\noindent\textit{Sub-critical}: This is a fission reaction which has failed to reach a state of being self-sustaining, and will eventually 
(if unaided) fizzle out
Often when dealing with nuclear transport codes, one wishes to answer the question of whether a system will be super- or sub-critical. 
This is, in large, the question we also seek to answer for some given initial conditions. 

\subsubsection{Neutron Transport}

We will build up the neutron transport code, describe its components, and then relate them to our project. We will largely 
follow the derivation given in \cite{mc-methods-neutron-transport-phd}. Where we come across a part of the derivation 
that relates to an assumption we make in our model, we will have an aside \textit{in italics} to emphasise the 
discrepancies between the standard neutron transport model and ours.

At the heart of neutron transport is a wish to describe how neutrons are distributed in some system. Let $N(\vec{r}, t)$ 
describe the density of neutrons in some medium, where $\vec{r} \in \R^3$ describes a position and $t \in \R$ a time. If we let 
$\vec{v}$ describe neutron velocity, 

To identify the particle density $N(\vec{r}, t)$, we need to take into account the particle velocities, $\vec{v}$, in the medium we're considering. 
We can describe the distribution of particles by the density function $n(\vec{r}, \vec{v}, t)$. The particle density can then be described
$$N(\vec{r}, t) = \int n(\vec{r}, \vec{v}, t) \dd^3 v$$

Normally we don't talk about particle velocities when dealing on atomic scales - instead we talk about orientation, and energy. 
If we let $\vec{\Omega} = \vec{v} / \norm{\vec{v}}$ be the unit vector which describes direction, then we can use the familiar $E = \frac{1}{2}mv^2$ 
to determine the particle's energy. Note then that the quantity
$$n(\vec{r}, E, \vec{\Omega}, t) \dd^3 \vec{r} \dd E \dd \vec{\Omega}$$
describes the number of particles in the volume $\dd^3 \vec{r}$ around the position $\vec{r}$ that have energy $E$ within $\dd E$, 
moving in direction $\vec{\Omega}$ within a direction change of $\dd \vec{\Omega}$. Note that we can describe the small differentials 
above in spherical coordinates, as we can reason that 
\begin{align*}
    \dd \vec{\Omega} &= \sin \theta \dd \theta \dd \varphi \\
    \dd^3 \vec{v} &= v^2 \dd v \sin \theta \dd \theta \dd \varphi \\
    \dd E &= mv \dd v
\end{align*}
If we take the energy version of our density function and make substitutions for above, we then get the relation
$$n(\vec{r}, E, \vec{\Omega}, t) = \frac{v}{m} n(\vec{r}, \vec{v}, t)$$
This relation will come in handy in a minute. \textit{Note here that this is a function of energy and angle. Recall from our 
model's assumptions that we specified our neutrons will be monoenergetic, and that we have an isotropic source. Thus, here, 
our value for $E$ won't be a variable, and similarly our direction vector will be distributed in an expected manner}.

Stepping aside, we'll consider what can affect the number of neutrons we have in our medium. The most obvious one is that there 
is some source of neutrons - we'll call this rate $q(\vec{r}, \vec{v}, t)$ (\textit{Recall we said we're assuming a single source of neutrons, which scatters isotropically, and only emits at the start 
of the simulation.}). Next, we know our neutrons can collide with the fissile material with 
some probability - we've already considered the possibility of fission reactions occurring, which is an example of the outcome 
of such a collision. This will require some expanding on, so for now we can just call this 
term $\left ( \pdv{n}{t} \right )_\text{coll}$. Then we know that neutron change can be described:
$$\dv{n(\vec{r}, \vec{v}, t)}{t} = \left ( \pdv{n}{t} \right )_\text{coll} + q(\vec{r}, \vec{v}, t) $$
where
\begin{align*}
    \dv{n}{t} &= \pdv{n}{t} + \pdv{\vec{r}}{t} \pdv{n}{\vec{r}} + \pdv{\vec{v}}{t} \pdv{n}{\vec{v}} \\
    &= \pdv{n}{t} + \vec{v} \nabla n + \frac{\vec{F}}{m} \pdv{n}{\vec{v}}
\end{align*}
and in the last line there we've used the familiar $F = ma = m \pdv{\vec{v}}{t}$. Thus, our neutron transport equation 
in its current state is:
$$\pdv{n}{t} + \vec{v} \nabla n + \frac{\vec{F}}{m} \pdv{n}{\vec{v}}= \left ( \pdv{n}{t} \right )_\text{coll} + q(\vec{r}, \vec{v}, t) $$
Note that the term $\frac{\vec{F}}{m} \pdv{n}{\vec{v}}$ is most often ignored when it comes to neutron transport as its effects are negligible.
So that slight simplification gives us:
$$\pdv{n}{t} + \vec{v} \nabla n = \left ( \pdv{n}{t} \right )_\text{coll} + q(\vec{r}, \vec{v}, t) $$
Where equality is taken in the physicist's sense. The next obvious step is to try and explain how collisions affect the neutron change. To do this, 
we need to talk about nuclear cross sections and collision kernels.

\noindent \textbf{Cross Sections}\\
When we talk about nuclear cross sections, we are talking about the number of interactions that occur for a particle in a given ``area''. 
Think you are driving a car through the streets of your home town. At any given point in time as you travel, there are any number of events 
happening around you - maybe a pedestrian is crossing the road, or a bird flies past, or maybe you hit another car while looking for 
things to notice. In some parts of town, say in the main street, it is more likely there will be things going on than if you were on the 
outskirts of town. Or maybe there is an event on in one area that makes it harder to focus on driving. A crude analogy, but 
in heart is comparable to what a cross section represents. The cross section, $\Sigma(\vec{r}, \vec{v})$, describes the average 
number of ``interactions'' that will occur for a particle if it is at position $\vec{r}$ (the location in town) travelling 
at speed $\vec{v}$ (however fast your are hooning). 

The dual to talking about a cross section is to talk about a particles ``mean free path'' instead. This is more intuitive, and 
effectively describes the average distance a particle would expect to travel before it has an ``event'' occur with another 
particle (i.e., a collision takes place). 

\textit{Note that one of our assumptions was that, when we come to do our simulation, all neutrons will have an event occur 
after moving. This is somewhat a lie as an assumption, and somewhat not. In reality we will effectively be drawing a randomly 
sampled distance travelled (that is informed by the material's nuclear cross section describing mean-free path), and assuming that 
a neutron has a collision after that distance. Our assumption would more accurately be represented as some form of time independence 
- whereby in a single pass of our simulation we may process a series of collisions simultaneously, though we would not 
necessarily expect these to actually occur simultaneously in a real life reaction, because of these varied distances travelled (among 
other reasons)}

\noindent \textbf{Scattering Probability}\\
Next, when an event does occur, i.e. a neutron collides with another particle, we need to describe how the two 
particles react to this. This is done using a ``scattering probability function'', which is an energy distribution function 
for the particle that is subject to the collision from the neutron. Let, then, $f(\vec{r}, \vec{v'} \to \vec{v})$ describe the 
probability that the secondary particle will be emitted with velocity $\vec{v}$, after the atom is struck by a neutron with 
velocity $\vec{v'}$ at position $\vec{r}$. 

\textit{In the neutron code, this distributino will be different for elastic and in-elastic scattering, and is, of course, 
a distribution. In our model however, we 1. are not considering in-elastic scattering, but 2. are also assuming a simple 
uniform distribution, as is used in \cite{simulations-neutron-transport} }

\noindent \textbf{Secondary Particle Count}\\
This one is easy to explain - when a collision event occurs (e.g. a fission reaction), what is the average number of secondary particles 
emitted by the collision event? Let $c(\vec{r}, \vec{v})$ describe such an average for a causal neutron of velocity $\vec{v}$, where 
the collision occurs at $\vec{r}$. This would describe, for example, the number of neutrons which come out of a fission reaction, and 
is what ties the speed of a colliding neutron with the energy going into the reaction. Intuitively we might expect an atom to react 
to a collision a bit differently if it a neutron is smashed into it as opposed to receiving a small bump (where ``small'' and 
``smashed into'' are taken to be their more scientific, scale-appropriate counterpart adjectives).

By convention we set this to $1$ for collisions which do not generate any secondary particles, such as absorption.

\textit{We stated in our assumptions that we would only consider $(n,2n)$ fission reactions for simplicity. Thus, this is 
actually just a piecewise function for us, there it is $2$ if we have a fission reaction, and otherwise is $1$.}

\noindent \textbf{Collision Kernel}\\
The collision kernel is what ties all the above together. It describes the average number of secondary particles of velocity $\vec{v}$ 
that are produced for every unit of distance travelled by a neutron with velocity $\vec{v'}$, at position $\vec{r}$. Then, combining 
what we've just reasoned about above, the collision kernel is given:
$$\Sigma(\vec{r}, \vec{v'} \to \vec{v}) = \Sigma(\vec{r}, \vec{v'}) c(\vec{r}, \vec{v'}) f(\vec{r}, \vec{v'} \to \vec{v})$$

Perhaps we can use a boxer as an analogy. Imagine you are standing in a boxing ring, but you have many opponents. Let this be a 
nightmare, and lets say you can't fight back against any punches you receive. How many punches might you expect to receive, 
as a function of where you are in the ring, and how fast will the teeth fly out of your mouth based on those punches? We might expect 
that if there are more opponents in the ring with you that you will lose more teeth (i.e., the number of secondary neutrons 
increases). Similarly, if some of these boxers are good at their job, then maybe the punches they throw will have a lot of energy. 
One might not wish to be hit by a punch with more energy, but perhaps in the process of contemplating whether it is something 
you enjoy or not you would notice that your teeth fly out of your mouth faster when you receive harder punches (i.e., 
neutrons which collide with an atom at higher speed will result in secondary neutrons being ejected at higher speeds). The
information the collision kernel captures is analogous to the number of teeth that fly from your mouth given a hard hitting 
punch from one of these ferocious foes, based on where you're standing in the ring. Though one does wonder if it is easier just 
to read the equation... 

\textit{When we come to our simulation we will essentially skip this as a necessity, and instead emulate the collisions themselves 
by random choices (the Monte Carlo method creeping in), and can handwave a lot of the complexity that a collision kernel 
seeks to represent away due to our assumptions of isotropic scattering and monoenergetic neutrons}

\noindent 
Now that we can describe the behaviour of our particles when a collision occurs, we can describe the collision term. 
For a neutron with velocity $\vec{v}$, its collision frequency can be described by $\norm{\vec{v}} \Sigma(\vec{r}, \vec{v})$. Thus, for
a population of neutrons with velocity $\vec{v}$ in a unit volume, the rate at which reactions occur is given by 
$$\norm{\vec{v}} \Sigma(\vec{r}, \vec{v}) n(\vec{r}, \vec{v}, t)$$
We are assuming that the particle that collides with the atom is ``lost'' (in reality it isn't lost, but is ejected again 
for example in the case of a fission reaction), and so for every collision with a particle $v$ the systme will lose $v \Sigma(\vec{r}, \vec{v}) n(\vec{r} \vec{v}, t)$ 
particles, but the system will gain however many particles come out of the collision event. Thus, the collision term can be summarised:
\begin{equation*}
    \left ( \pdv{n}{t} \right )_{\text{coll}} = \overbrace{\int v' \Sigma(\vec{r}, \vec{v'} \to \vec{v}) n(\vec{r}, \vec{v'}, t) \dd^3 v'}^{\text{Neutrons gained from all possible collision events}} - \overbrace{v \Sigma(\vec{r}, \vec{v}) n(\vec{r}, \vec{v}, t)}^{\text{Neutrons which participate in a collision}} 
\end{equation*}
Substituting this back into our transport equation:
\begin{align*}
    \pdv{n}{t} + \vec{v} \nabla n &= \left ( \pdv{n}{t} \right )_\text{coll} + q(\vec{r}, \vec{v}, t) \\
    \pdv{n}{t} + \vec{v} \nabla n + v \Sigma(\vec{r}, \vec{v}) n(\vec{r}, \vec{v}, t) &= \int v' \Sigma(\vec{r}, \vec{v'} \to \vec{v}) n(\vec{r}, \vec{v'}, t) \dd^3 v' + q(\vec{r}, \vec{v}, t)
\end{align*}
And were done! This is (a) general neutron transport equation. In deriving this we've highlighted a few key points, as in our simulations, 
we don't actually seek to implement this equation directly. Instead, we implement components that were used to build the equation up. Next we'll discuss 
what we will simulate, and how our assumptions tie into modifications to the above.

\noindent \textbf{Energy Representations} \\
Notably, in the above derivation, we've kept everything in terms of velocity. However, as we noted at the start of the derivation, 
it is possible to instead express those same quantities in terms of a direction vector $\vec{\Omega}$, and particle energy, $E$.
In that spirit, we can express our functions in terms of these similarly:
\begin{align*}
    \Sigma(\vec{r}, \vec{v'} \to \vec{v}) &\iff \Sigma(\vec{r}, E' \to E, \vec{\Omega'} \to \vec{\Omega}) \\
    f(\vec{r}, \vec{v'} \to \vec{v}) &\iff f(\vec{r}, E' \to E, \vec{\Omega'} \to \vec{\Omega})
\end{align*}

\subsection{Methodology}

In our experiment, we don't wish to directly solve the neutron transport code we've just derived. Instead, we seek to model components of it, 
and run a Monte Carlo style simulation of it. We are implementing a particle simulation, and so will start by defining the properties 
of our system. We will have a sphere of varying radius, though will be approximately on the order of $10^{-10}\text{cm}$. The reason for 
this is explained in the limitations section. 

Our simulation process is described below. For every simulation cycle:
\begin{enumerate}
    \item For every neutron present in the simulation:
    \begin{enumerate}
        \item If this neutron is outside the medium, or it has been absorbed, skip processing it
        \item Randomly sample this neutron's next travel direction and distance
        \item Determine its new $(x,y,z)$ position
        \item Check if the neutron has moved to be outside the medium. If it has, mark it as having escaped and return to (1)
        \item Randomly choose an event to occur, with weight to optinos given by nuclear cross section data:
        \begin{itemize}
            \item Elastic scattering collision. Then continue (neutron will be ``scatterred'' at start of next simulation cycle).
            \item Capture collision. Mark the neutron as absorbed, and continue.
            \item Fission collision. Generate a new neutron at this position, and continue.
        \end{itemize}
    \end{enumerate}
    \item Collect data about neutrons present, generated, escaped, and captured
\end{enumerate}

The reason this can be referred to as a ``Monte Carlo Method'' is seen when we consider the stochastic behaviour that we introduce into 
the system. When we decide a neutron's next direction to move in, we randomly sample its direction and distance (with experimental 
data to inform this sampling). Similarly, using data, we determine 

Let $(L, \theta, \phi)$ describe a neutron's next direction to move and the length of that movement, and let $\xi$ 
denote a uniformly sampled random variable. In our assumptions, we stated 
that we are assuming isotropic scattering - thus, $\phi = 2\pi \xi$ (this being the off axis scattering angle). 
For the scattering angle in the plane of collision, however, we cannot simply assume a uniform distribution. 
Often we take a distribution such that $\mu = \sin \theta = -1 + 2\xi$ \cite{mc-methods-neutron-transport-phd}. Thus, 
to determine $\theta$, we have $\theta = \asin \left (-1 + 2\xi \right )$. Normally this angle would affect the energy loss, 
however because we are assuming that all our neutrons have the same energy (and no in-elastic collisions occur), this 
energy calculation is something we ignore (this is what we refer to when we say we're breaking the laws of 
thermodynamics in our limitations section). We then know which direction our scattered neutron will travel - next we 
need to know how far it will move.

As was described in the previous section, the nuclear cross section describes, roughly, how often we would expect a specific
collision event to occur. Its dual statement is concerns ``mean-free paths'', as we have also described. In our simulation, 
we can randomly sample this mean-free path value to determine how far our neutron may travel. We will justify the relation we 
get, however, and draw this conclusion from \cite{simulations-neutron-transport}. First, let $P(L)$ describe the 
probability that a particle will move $L$ units of distance without making a collision. Then
$$P(L) = \frac{\text{number of particles that move to depth of L}}{\text{number of trials}} = e^{-\Sigma_t (E) L}$$
where $\Sigma_t (E)$ describes the total nuclear cross section (the sum of individual cross sections) for a particle of energy $E$.
Recall again that this is a constant for us, as we assume monoenergetic neutrons. This is a fairly standard probability 
distribution function, and so then we can describe the probability of a neutron travelling some path up to $L$ long as 
the CDF of this:
$$F(L) = 1 - P(L) = 1 - N_0 e^{-\Sigma_t(E) L}$$
where $N_0$ is the number of trials. If we wish to sample the length $L$ to first collision (as we do in our simulation, note), then 
letting this be $\xi$ again, we have $F(L) = \xi$, and so:
\begin{align*}
    \xi = F(L) &= 1 - e^{-\Sigma_t(D) L} \\
    1 - \xi &= e^{-\Sigma_t(E) L} \\
    \iff L &= -\frac{1}{\Sigma_t(E)} \ln(1 - \xi)
    \shortintertext{But noting that $\xi$ is uniformly distributed, $1 - \xi$ is equivalent to $\xi$, so we can just write}
    L &= -\frac{1}{\Sigma_t(E)} \ln(\xi)
\end{align*}
Then we only hvae to reason about this strange $-\frac{1}{\Sigma_t(E)}$ term we have. We have a relation between mean-free path and 
collision kernels, hwich states that 
$$\lambda = (n\sigma)^{-1}$$
where $\lambda$ is an expected path distance for a specific cross section property, and $\sigma$ describes that cross section 
(these are referred to as micro cross-sections, and relate directly to their macroscopic counterparts). Here, $n$ is the 
number of neutrons in the volume being considered (which in our case, is a sphere, so $v = \frac{4}{3}\pi r^3$, and $N$ is 
the number of neutrons present in our simulation). If we let $\sigma_{sc}, \sigma_{cp}$ and $\sigma_{f}$ describe the cross 
sections for our elastic scattering collisions, capture collisions, and fission collisions respectively, then we can describe 
the total cross section as $\Sigma_t = \sigma_{sc} + \sigma_{cp} + \sigma_{f}$. Then we can determine $\lambda_t = (n \cdot \Sigma_t)^{-1}$.
Putting all this together, we can determine the length of the path our neutron will travel to be:
$$L = -\lambda_t \ln(\xi)$$
So to summarise, our randomly sampled (Monte Carlo method!) next (unit vector) direction and distance is given by:
\begin{align*}
    L &= -\lambda_t \ln(\xi_1) \\
    \theta &= \asin(-1 + 2 \cdot \xi_2) \\
    \phi &= 2 \pi \xi_3
\end{align*}
where $\xi_1, \xi_2, \xi_3$ are all uniformly distributed random variables on $[0,1]$. The last thing to do is to determine the actual 
differential for our neutron. This is a straightforward 3D vector calculation, and is given:
\begin{align*}
    \dd x &= L \cdot \cos(\theta) \cdot \cos(\phi) \\
    \dd y &= L \cdot \cos(\theta) \cdot \sin(\phi) \\
    \dd z &= L \cdot \sin(\theta) 
\end{align*}

Our decision process for which collision event occurs is relatively simple. We distribute probabilities of each event occurring 
according to the proportion of the total cross section they represent, and use a random variable $\xi$ to decide which one 
occurs.

When we talk about these corss section values, we use Barns units. $1 \text{barn} = 10^{-24}\text{cm}^2$. The relevant cross 
sections for Uranium-235 and Plutonium-239 are given in table \ref{cross-sections}. The important values 
for us are that of the thermal neutrons (as we are assuming monenergetic neutrons with energy at that level). 
To highlight the effect that the energy of a neutron has on its ability to engage in a fission reaction however, 
we include the same values for neutrons travelling at considerably faster (relativistic) energies for contrast. A 
quick observation would note that fission reactions are generally, maybe unintuitively, less likely to occur when the bombarding 
neutrons are too fast -- though the physical explanation for this delves into quantum theory, and is out of scope for this project.

\begin{table}[h!]
    \centering
    \begin{tabular}{l|lll|}
    \cline{2-4}
                                        & \multicolumn{3}{l|}{Thermal (Barns)} \\ \hline
    \multicolumn{1}{|l|}{Material} & \multicolumn{1}{l|}{Scatter} & \multicolumn{1}{l|}{Capture} & Fission \\ \hline
    \multicolumn{1}{|l|}{Uranium-235}   & 10        & 99          & 583        \\ \cline{1-1}
    \multicolumn{1}{|l|}{Plutonium-239} & 8         & 269         & 748        \\ \hline
    \multicolumn{1}{|l|}{}              & \multicolumn{3}{l|}{Fast (Barns)}    \\ \hline
    \multicolumn{1}{|l|}{Uranium-235}   & 5         & 0.07        & 0.3        \\ \cline{1-1}
    \multicolumn{1}{|l|}{Plutonium-239} & 5         & 0.05        & 2          \\ \hline
    \end{tabular}
    \caption{Averaged cross section data for Uranium-235 and Plutonium-239. Here, a ``thermal'' neutron is taken to be travelling at $2200$m/s, 
    and the values provides are in Barns ($10^{-24} \text{cm}^2$). This data retrieved from the JEFF nuclear data repository \cite{collision-kernel-data}.}
    \label{cross-sections}
\end{table}\newpage


\subsubsection{Alternative Approaches}

\noindent \textbf{Fluid Dynamics} \\
There are simplified models, such as the one described in \cite{los-alamos-primer}, which are more akin to a fluid dynamics approach. 
In a fluid dynamics approach, instead of tracking the behaviour of individual particles, you model the population of neutrons as a fluid 
which moves about within the medium. The behaviour of the neutrons can then be described by balance equations which act as flow in and out 
of the system. This method has the benefit of being considerably more computationally feasible than a particle simulation, but sacrifices accuracy 
in its representation of a system -- particularly given the nature of a chain reaction somewhat necessitating the tracking of individual particles due 
to it being dependent on energy of individual particles and their orientation.

\noindent \textbf{Monte Carlo Integration} \\
Another method which is keeping in the spirit of working with Monte Carlo related methods, would be to evaluate the given integral directly.
Doing so would require functions which approximate the required behaviours for a given material (for example, $f(\vec{r}, \vec{v'} \to \vec{v})$), 
though these approximations are likely not impossible given the abundance of experimental data available (one could also simply use data as a sort of 
reference table instead of having an analytic function to represent it). If you have a way to functionally describe all components of the equation, then 
to save ones sanity, instead of attempting to find an antiderivative for that function you could use Monte Carlo integration to evaluate it. 
Given the inherent dimensionality of the problem and the potential for sensitivity to parameters, it is possible this approach will be quite 
computationally expensive however, and so perhaps should be done with aid of hardware that surpasses the ability of a ThinkPad T440p.

\subsubsection{Limitations of our Model}

We might note that, in building our model, we've made a great deal many assumptions about the properties of our system. 
This would be an astute observation, and in fact we should make no claim to precision in our model whatsoever. Instead of 
determining a precise value for critical density, or reliably measuring properties surrounding the situations we've constructed, 
we instead set the goal of merely observing the trends we would expect to find in such a system. What we have is not so much 
a spherical cow model of a bomb, as it is a cow-shaped model of a bomb.

Why do we place such a restriction on ourselves? This is because the assumptions we make limit our model considerably. 
For example, our assumption that all neutrons in the medium have the same kinetic energy takes away from the accuracy of our model, 
in that the system breaks the laws of thermodynamics -- our collisions result in no energy lost (in way 
of a neutron being slowed), but themselves create energy (a new neutron with the same energy can be produced). 

Similarly our assumption that there is not an external source of neutrons could be considered an inaccuracy. It is 
possible to construct your aparatus so as to limit the presence of external neutrons past what we inject into our experiment, 
however it is foolish to assume one could entirely rid a system of external influence -- this is especially significant for our 
simulation when you consider the scale we've limited ourselves to (a note on which is below).

We would be remiss to not note the computational restriction we face as well. Assume that we could rid ourselves of 
the above assumptions and have a perfectly physical model of the physics of our system. Then even then, we could not trust 
our results, as to do so would require the simulation of a number of neutrons on order far greater than what my laptop is 
capable of handling -- at the point you wish to run a simulation like that, you should be asking for super computer time. As 
such, we have a significantly reduced quantity of neutrons present in our system. We somewhat account for this by scaling our 
system to accommodate for this, in that instead of dealing with a bomb at a regular diameter such as the $17.32\text{cm}$ radius 
Los Alamos Trinity test, we reduce the scale of our system to be physically infeasible itself as well.

In short, our model should be considered extremely unphysical, and is more of an academic interest than an actual 
experiment we would seek results from. Nonetheless, it is fun to see a miniature nuclear reaction!