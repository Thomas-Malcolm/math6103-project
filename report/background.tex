

\section{Background}

\subsection{What is the purpose of neutron transport codes?}

As we alluded to in the introduction, neutrons play a crucial role in engineering problems relating to nuclear energy production 
and control. Neutrons are the driving force in fission reactions (technical justification for which will be covered in the theory section 
), and so it is important to understand the dynamics of a population of neutrons within a fissile material. This is where neutron transport comes into play. 
Neutron transport studies concern the behaviour of neutrons within some medium, highlighting the number of neutrons present, their movement, 
and their interaction with the medium. 

This knowledge isn't solely limited to nuclear fission however, nor is it constrained to weapons development.
We'll highlight a couple particularly interesting (and relevant) applications for neutron transport codes, though should emphasise 
that this list is by no means exhaustive.

\noindent\textbf{Nuclear Weapons}\\
As highlighted in the introduction, the development of nuclear weapons during World War Two was a significant motivating 
factor for early study of neutron behaviour, and their relation to nuclear chain reactions \cite{los-alamos-primer}. 
It was only as a result of development at Los Alamos that neutron codes came into the forefront of academia, and their 
application is seen directly. The most obvious use case for neutron transport codes in nuclear weapons development 
(and what we seek to observe in this project) is using them to determine the conditions required for a 
chain reaction to be super-critical - that is, for the reaction to be self-sustaining in a way that grows to consume 
all available fissile material, otherwise known as an explosion.

\noindent\textbf{Nuclear Fission Reactors}\\
The natural lead on question from ``can we exploit the energy released in a fission reaction to build a bomb'', is ``can we exploit 
the energy released in a fission reaction to power our homes''? A bit more wholesome a question, the answer as we know is 
of course yes. Neutron transport codes (and more generally readiation transport codes, which are just more abstract 
particle transport codes) are used in a similar manner to how they are in nuclear weapons development, though with 
a different aim. Whereas for nuclear weapons we want to see our chain reaction reach super-critical state, we do not 
want this for our reactors - it would be counterproductive to spend much time building a reactor, only for it 
to suddenly not exist. Thus we use transport codes here to determine what conditions are required to keep 
chained fission reactions at a controllable level (i.e., sub-critical), while still being able to harness energy from it.

\noindent\textbf{Radiation Shielding}\\
Being inherently a particle simulation that considers radioactive materials and their interactions with their environment, 
neutron transport codes see a direct application in providing consult on how best to design reactors to safeguard 
us from its effects as well. For example, one can show that water is an excellent radiation protection mechanism by running 
a simulation which describes a region of water surrounding a fissile material that interacts with a population of neutrons. 
You can measure the (simulated) radiation levels outside the water region. This kind of simulation is more 
commonly known as a neutron diffusion, and is a simplificatoin of neutron transport.

\noindent\textbf{Particle Accelerator Design}\\
Particle accelerators have been using transport codes to simulate the dynamics of interactions between nuclei 
and sub-atomic particles \cite{particle-accelerator-neutron-codes}.

\noindent\textbf{Fusion Reactor Design}\\
Radiation transport codes are used to 
determine radiation amounts, and general particle interactions, inside a fusion reactor \cite{fusion-neutron-codes}.
These help inform design decisions, and provide a way to test components of a reactor without the expensive and time 
consuming process of constructing one (see for evidence, ITER).

\subsection{What work has been done in the field?}

Since the 1940s, neutron transport codes have seen wide reaching applications, as we've highlighted above. However this came about only 
after years of intensive research in both physics and mathematics, and has itself been the generator of new concepts in both fields. 
Here we'll provide a briefly highlight the outcomes of neutron transport research in the past, and in the modern day.

\noindent\textbf{Historical Work}\\
We have already covered much of the historical work done and what led to interest being developed in neutron transport codes, and so 
will not repeat ourselves here. However, we will make note on one other core idea which emerged from the nuclear weapons development 
at Los Alamos shortly after World War Two, which is the work done by John von Neumann and Stanislaw Ulam in developing 
Monte Carlo methods. In fact, the work they did was directly borne from a wish to study neutron chain reactions. 
The term ``Monte Carlo Method'' can be used to describe a broad class of solutions to problems, but in large they all share 
the characteristic that they depend on some stochastic process for their simulation. In the theory section we will cover specifically 
what a Monte Carlo Method means to us, and how we will use them, but for now we will just note their significance. The discovery 
of Monte Carlo methods came at a time where computers were beginning to become both more powerful and available, which 
helped propell them to academic stardom, as they became the subject of intense research. Since then, much work has been done 
on developing lead on processes and applying them to various fields, where today the study of Monte Carlo methods can be considered 
its own sub-field of statistics, with its own class of problems and applications -- and only a memory of its beginnings 
as a tool for nuclear weapons development \cite{dirk-monte-carlo}.


\noindent\textbf{Modern Work}\\
More modern research largely concerns numerical methods for computing properties around neutron transport for increasingly complex 
geometries and chemical compositions -- essentially a problem of quantity and efficiency. Somewhat understandably however, 
much of the work done in this field is either classified or commercial in confidence, which is to say source code for industry 
standard simulations is next to impossible to find, and even getting access to software for running your own simulations 
requires permission from the owners and a large sum of money. Nevertheless, one such well respected code is the ``Monte Carlo N-Particle 
Transport Code'' (MCNP). It is capable of very general simulations, and is used in simulations 
for nuclear reactors, among other things. Perhaps expectedly, it is developed by Los Alamos \cite{mcnp-code}. 

Where in this project we simplify our problem considerably to be for a uniform neutron energy, spherical geometry, 
discretised time neutron particle simulation that is capable of handling only a few thousand particles, the MCMP code is 
much more robust. It is able to handle neutrons that have a specifies energy distribution, can run the simulation 
for a prescribed geometry (i.e., is not confined to that of a sphere), and can handle many orders larger than a few thousand 
neutrons. Additionally, it provides a wealth of diagnostics information and features for simulating scenarios. For example, 
it is able to handle the existence of neutron shields -- something we propose as an extension to this project in our conclusions. 
It is also malleable enough to not just be useful for neutron particle simulations, but can also handle electron transport as well. 
To refrain from fawning over a piece of technology further, it is a highly capable industry standard code for neutron transport, and this project we 
have completed here could itself be viewed as a toy version of MCNP.