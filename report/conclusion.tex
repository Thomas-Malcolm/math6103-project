\section{Conclusions}

\subsection{What did we achieve?}

\subsection{Future work?}


\textbf{Neutron Reflector} \\
When it comes to nuclear reactors, one goal of your system is to keep the neutron count in equilibrium so as to ensure a sustained 
reaction that doesn't grow too quickly and lead to super-critical conditions (i.e., an explosion). However, there is a case 
where you would want this rapid growth in reactions - in the case that you want an explosion! Whereas nuclear reactors have 
devices designed to regulate neutron energy and count, a nuclear bomb has the opposite goal. However, one of the larger 
causes for loss of neutrons is that they simply leave the medium you are trying to contain them within. As such, one 
method often used (which was invented at Los Alomos for the Trinity test) is to have a ``neutron reflector''. This is an 
outer casing around your bomb which biases the reflection angle when it comes to elastic scattering collision events for neutrons 
operating within the bounds of the bomb medium. This decreases the loss of neutrons to the environment, and thus maintains 
density of neutrons for longer, making it easier for a sustained and super-critical fission reaction to develop. One such extension 
could be then to include support for such a structure into the simulation, and to observe how it affects the critical density of both 
isotopes. 

\noindent\textbf{Other Fission Reactions} \\
Currently in the simulation, if a neutron's collision is deemed to produce a fission event, we only support the case that one neutron 
is ejected from the atom our neutron has collided with. This is referred to as an (n,2n) reaction. However, it is possible 
for more outputs -- an (n,3n) reaction is possible for example, where two excess neutrons are ejected from the target atom. This would 
increase the number of neutrons present in the medium, and is more physically accurate. 

\noindent\textbf{More Precise Scattering Variants} \\
Currently only in-elastic scattering is supported, however in real kinetic dynamics, this is infeasible. When a neutron collides 
with an atom, it should be affected by the momentum of the atom, instead of randomly picking a direction to head in next. Similarly, 
for cases of fission events, the resulting neutron should be affected by the 

\noindent\textbf{Neutron Energy Distributions} \\
In simplifying our model to assume a uniform, averaged neutron energy, we ignore many effects. For example, there is a significant difference 
between mean free-paths for thermal and fast neutrons, and each have their own purpose. For example, in a nuclear reactor you wish to have 
more thermal neutrons as they caise fission reactions over longer time scales, whereas fast neutrons will result in more fission events 
in shorter periods. Some analytic approximations to energy distributions for neutrons within medium are available, and these could be incorporated 
into our model to more accurately represent the behaviour of neutrons. This would well be accompanied by a more precise implementation 
of scattering methods as described above, as they are also somewhat dependent on energies of participating neutrons.

\noindent\textbf{Reaction Medium Geometries} \\
We could also support different geometries. Currently we assume a sphere, which is quite simplistic. Some more academic 
structures such as a cube would not be difficult to implement and provide some interesting analytics, however some 
other geometries such as a cylinder could be useful for simulations on reactor cores for example.
