

\section{Introduction}

On July 16, 1945, the scientists working at Los Alomos on the Manhatten Project bared witness to the culmination of years 
of research and innovation -- the successful execution of the Trinity test. The discovery that bombarding a nucleus with neutrons 
could result in an atom being split into two smaller atoms, and in the process release a relatively large amount of energy, was 
a significant moment in the history of mankind. The subsequent realisation that such a process could result in a ``knock-on'' effect, 
triggering more atoms to be split, releasing even more energy, was the next significant moment. These simple ideas have themselves 
provided us with so much death, destruction and fear, while simultaneously giving us power and crucial research insights that in the 
sweet irony that only life can provide, also helps save lives, in fields such as medical imaging.



\subsection{Project Goals}

This project is about creating a toy box through which to demonstrate the physics the scientists at Los Alomons sought to exploit 
in their development of the first atomic bomb. Though we should note that, while a nuclear detonation device is the original 
inspiration for the work they did (and thus this project), the utility of the work is not limited to just that. Through simple 
tweaking of the code one could use it as a diffusion simulation, for example, to see how neutrons disperse through a water medium. But 
we digress. 

In this project we will implement a particle simulation for neutrons in a constrained medium, and track the behaviour of this population 
as they interact with their surroundings. As we will discuss, neutrons are what drive fission reactions in fissile materials - so it 
stands to reason that to have more neutrons would be to have more chance of having a fission reaction occur, and thus more chance of a, 
to be scientific, ``large boom''. However, merely having neutrons present is not sufficient for ensuring that a sustained fission reaction occurs 
- after all, neutrons are always 'present' around us, and yet Uranium doesn't seem to spontaneously combust in nature (or so the German's would have 
us believe). So the question becomes: what do we need to do to ensure that it does? 

In the spirit of keeping with the Los Alomos designs, we will assume a perfectly spherical geometry for our medium. 
To simplify our work, we will also assume our ``bomb'' is entirely composed of a single isotope mass of particles, which 
we will take to either be Uranium-235 or Plutonium-239 (for their fissile nature, which is to say, their willingness 
to undergo a fission reaction under certain conditions, which we will discuss). For our neutrons, we will assume that 
our population of particles are travelling with the same energy, and that they are not travelling so fast as to require relativistic 
adjustments (though this actually plays into our favour due to quantum effects which affect the fissile efficacy of materials 
when bombarded with neutrons travelling at such high speeds). All these assumptions we for now state without explanation, though 
will describe later in this report.

The results we seek are to see that, adjusted for computational limitations, there is a density (unique to each isotope) 
which acts as a boundary separating whether a fission reaction will be sustained or not. We will compare our results 
to expected trends, and discuss what our simulations suggest for the materials.